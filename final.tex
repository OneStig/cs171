\documentclass[a4paper,10pt]{article}
\usepackage[utf8]{inputenc}
\usepackage{amsmath, amssymb}
\usepackage{tcolorbox}
\usepackage[margin=0.3in]{geometry}
\usepackage{multicol}

\usepackage{enumitem}
\setlist[itemize]{itemsep=0pt, parsep=0pt, left=2pt, topsep=0pt, labelsep=2pt, label=$\cdot$}

\makeatletter
\g@addto@macro\normalsize{
    \setlength{\abovedisplayskip}{2pt}
    \setlength{\belowdisplayskip}{2pt}
    \setlength{\abovedisplayshortskip}{2pt}
    \setlength{\belowdisplayshortskip}{2pt}
}
\makeatother

\pagestyle{empty}

\tcbset{
    module/.style={
        colback=white,
        colframe=black,
        fonttitle=\bfseries\scriptsize,
        fontupper=\scriptsize,
        boxrule=0.4pt,
        arc=1mm,
        outer arc=1mm,
        colbacktitle=gray!20,
        coltitle=black,
        boxsep=0.5mm,
        left=0.5mm, right=0.5mm, top=0.5mm, bottom=0.5mm,
    },
}

\begin{document}

\vspace{-1em}
\begin{center}
    \textbf{\footnotesize CS 171 F24 Final Exam Reference Sheet} \\[-0.5em]
    \textit{\scriptsize Created by Steven He}
\end{center}

\vspace{-1em}

\begin{multicols*}{2}

\begin{tcolorbox}[title=Uninformed Search, module]
    A \textbf{state} represents particular configuration of environment. \textbf{Nodes} in search tree can encode details like state, parent, action, path cost, depth, etc.

    \textbf{Tree Search} doesn't remember visited nodes: slow search, low memory
    \textbf{Graph Search} remembers visited nodes: fast search, high memory

    Basic search scheme maintains explored and frontier nodes. Process the frontier in some order based on some policy/search strategy, marking as visited if graph search, then expanding node.

    \textbf{Strategies evaluted by:}
    \begin{itemize}
        \item \textit{completeness} - does it guarantee a solution if one exists
        \item \textit{optimality} - does it guarantee min-cost solution
        \item \textit{time/space complexity} - \# of nodes generated / max \# of nodes in memory. In terms of max branching factor $b$, depth of min-cost solution $d$, maximum depth of state space $m$.
    \end{itemize}

    \textbf{Uninformed search strategies:}
    \begin{itemize}
        \item \textit{Breadth-first Search} - Frontier is FIFO queue. Goal test before pushing to queue. Always complete; only optimal if cost is monotonic non-decreasing function of depth.
        \item \textit{Uniform-cost Search} - Frontier is priority queue, ordered by total path cost $g(n)$. Goal test after popping from queue. Complete if $b$ is finite; optimal if all step costs $\ge \epsilon > 0$ (non-zero).
        \item \textit{Depth-first Search} - Frontier is LIFO stack. Defined as Depth-limited search with limit $l = \infty$. Goal test after pop from stack. Incomplete if depth of states is infinite; not optimal.
        \item \textit{Iterative Deepening Search} - Perform Depth-limited search, while iteratively increasing depth limit. Memory advantage of DFS, completeness \& optimality conditions of BFS.
    \end{itemize}

\end{tcolorbox}


\begin{tcolorbox}[title=Heuristic Search, module]
    \textbf{todo}
\end{tcolorbox}

\begin{tcolorbox}[title=Game Search, module]
    \textbf{todo}
\end{tcolorbox}

\begin{tcolorbox}[title=Constraint Satisfaction, module]
    \textbf{todo}
\end{tcolorbox}

\begin{tcolorbox}[title=Logic, module]
    \textbf{todo}
\end{tcolorbox}

\begin{tcolorbox}[title={Probability \& Uncertainty, Bayesian Networks}, module]
    \textbf{todo}
\end{tcolorbox}

\begin{tcolorbox}[title={Intro to ML, Linear Regression, kNN}, module]
    \textbf{todo}
\end{tcolorbox}

\begin{tcolorbox}[title={Decision Trees and Neural Networks}, module]
    \textbf{todo}
\end{tcolorbox}

\begin{tcolorbox}[title={Reinforcement Learning}, module]
    \textbf{Markov Property} - Future is independent of past given present:
    \[
        \mathbb{P}[S_{t + 1} | S_t] = \mathbb{P}[S_{t + 1} | S_1, \ldots, S_t] \text{ where } S_t \text{ is state at time } t
    \]

    We use matrix $\mathcal{P}$ to define transition property from state $s$ to $s'$, denoted as probability in row $s$, column $s'$.
    \[
        \mathcal{P} =
        \begin{bmatrix}
            \mathcal{P}_{11} & \ldots & \mathcal{P}_{1n} \\
            \vdots && \vdots \\
            \mathcal{P}_{n1} & \ldots & \mathcal{P}_{nn} \\
        \end{bmatrix}
        , \mathcal{P}_{ss'} = \mathbb{P}[S_{t+1} = s' | S_t = s]
    \]

    \textbf{Markov Process/Chain} - Sequence of states $S_1, S_2, \ldots$ satisfying Markov property. Formally defined as tuple $\langle \mathcal{S}, \mathcal{P} \rangle$ i.e. (set of states, prob matrix)

    \textbf{Episode} - Some sequence of traveresed states in a MP

    \textbf{Markov Reward Process} - Give states in MP some reward. We ``gain'' reward $R_{t + 1}$ when transitioning from states $S_t \to S_{t + 1}$

    placeholder intermediary stuff goes here

    \[
        \begin{tabular}{|c|c|c|}
            \hline
            & \textbf{Evaluate Policy, $\pi$} & \textbf{Find Best Policy, $\pi^*$} \\
            \hline
            \textbf{MDP Known} & Policy Evaluation & Policy/Value Iteration \\
            \textbf{(Planning probs)} && \\
            \hline
            \textbf{MDP Unknown} & MC and TD Learning & Q-Learning \\
            \hline
        \end{tabular}
    \]

\end{tcolorbox}

\end{multicols*}

\end{document}
